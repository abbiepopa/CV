% LaTeX resume using res.cls
\UseRawInputEncoding
\documentclass[line,margin,10pt]{res} 
\usepackage{enumitem}
\usepackage{hyperref}
\oddsidemargin -0.3in
\textwidth 5.75in

\begin{document}

\name{Abbie M. Popa, Ph.D.}
 
\begin{resume}
\moveleft 0.5\hoffset\centerline 
{\hyperref[abbiepopa@gmail.com]{abbiepopa@gmail.com} \quad \quad \quad \quad  \quad \quad \quad \quad \quad \quad \quad \quad \quad \quad \quad \quad \quad \quad \quad \quad \quad\quad \quad \quad \quad \quad \quad \quad \quad \quad \quad \quad  \quad \quad \quad   401-440-5228}
\moveleft 0.5\hoffset\centerline 
{\hspace{0.05cm} \hyperref[linkedin.com/in/abbiepopa]{linkedin.com/in/abbiepopa} \quad \quad \quad \quad \quad \quad \quad \quad \quad \quad \quad\quad \quad \quad \quad \quad \quad \quad \quad \quad \quad \quad  \quad \quad \quad  \quad \quad \quad   \quad \hyperref[github.com/abbiepopa]{github.com/abbiepopa}}
 
\section{EDUCATION} 
\textbf{Ph.D.}, University of California at Davis \hfill (September 2012 - June 2018)\\
Major: Cognitive Neuroscience \hfill
Data Science Initiative Affiliation\\
Dissertation: Behavior and EEG Testing of Teen Anxiety \\
%\emph{Thesis Topic}: Anxiety Impacts Attentional and Inhibitory Control in Adolescence\\
\textbf{Honors Sc.B.}, Brown University \hfill (September 2006-December 2010)\\
                % \sl will be bold italic in New Century Schoolbook (or
	        % any postscript font) and just slanted in
		% Computer Modern (default) font
                Major: Cognitive Neuroscience

%\section{DISSERTATION}
%\textbf{Behavior and EEG Testing of Teen Anxiety}\\
%Mentor: Dr. Tony J. Simon \hfill June 2013 - June 2018\\
%A study of 23 adolescents with anxiety disorders and 38 typical controls examining attentional and inhibitory control using event-related potentials, heart rate, and behavioral data. Participants completed four tasks and self and parent reports of daily function.  Data analysis included ICA and mixed effects linear models. 

\section{TECHNICAL EXPERIENCE}

\textbf{The Data Institute at the University of San Francisco} \hfill San Francisco, CA \\
{\sl Data Science Postdoctoral Fellow} \hfill August 2018 - Present
\begin{itemize} \itemsep -2pt
\item Using self-supervised learning to generate feature embeddings that describe nodes in brain networks from patients with schizophrenia and healthy controls
\item Using machine learning techniques including random forests and support vector machines to classify non-linear features in EEG from infants who were born preterm
\item Contributing to a collaborative reading and practice group on Reinforcement Learning
\item Consulting on collaborations with 4 private sector companies
\item Instructed and developed an Introduction to Data Science class for 40 undergraduates
\end{itemize}


\textbf{UC Davis MIND Institute} \hfill Sacramento, CA \\
{\sl PhD Researcher} \hfill September 2012 - June 2018
\begin{itemize} \itemsep -2pt
\item Used techniques including the jack-knife method to derive signal from event-related potentials in electroencephalography (EEG)
\item Developed 6 child-friendly computerized behavior tests (disguised as games)
\item Used k-means clustering to classify children as "copers" or "strugglers" based on behavioral, eye-tracking, and self-report measures
\item Used ICA to isolate brain activity from noise in EEG data
\item Used mixed effects linear modeling to classify participants behavior over time
\item Used linear regression to correlate trajectories of brain development and children's outcomes in a large (approximately 500 GB) MRI dataset
\item Programmed data analyses and visual stimuli using R, Python, and Matlab
\item Trained and mentored 6 junior research assistants and 7 volunteer interns
\end{itemize}

\textbf{Davis Incubator Group} \hfill Student organized group at UC Davis\\
%Davis, CA \hfill January 2016 - Present\\
{\sl President} \hfill September 2016 - present\\
{\sl Member} \hfill January - August 2016 
\begin{itemize} \itemsep -2pt
\item Completed online coursework in python, SQL, and machine learning
\item Completed two collaborative Kaggle image classification challenges using skimage, TensorFlow, tflearn, keras, OpenCV and PIL on an AWS machine to efficiently localize and classify images through convolutional neural networks for datasets up to 100 GB.
\item Completed a collaborative Driven Data competition using Pandas and sklearn to finish in the top 10\% of competitors
\item Developed individual projects using tools including sklearn in Python, and nnet in R
\item Organized and scheduled meetings for a group of 6-8 data scientists to practice coding, machine learning, and share data science skills
\end{itemize}

\textbf{UC Davis Data Science Initiative} \hfill  Davis, CA \\
{\sl Affiliate} \hfill April 2016 - present
\begin{itemize} \itemsep -2pt
\item Contributed to a collaborative reading group on {\sl Think Python} where we read and discussed the chapters and implemented the exercises as well as elaborated on best practices
\item Implementing {\sl An Introduction to Statistical Learning} collaboratively using Python and packages including statsmodels, scipy, numpy, pandas, sklearn, matplotlib, and seaborn
\item Consulted with members of the UC Davis community on data science problems from twitter scraping to genomics as part of team un-seminars
\end{itemize}
 
  \section{TEACHING\\EXPERIENCE} 
  
   \textbf{Software Carpentry} \\ 
 {\sl Certified Instructor for bash, git, R, and Python} \hfill May 2018 - Present
 \begin{itemize}\itemsep -2pt
 \item Trained and certified in core pedagogical concepts and evidence based teaching practices
 \item Certified to teach bash, git, R, and python to scientists at workshops 
 \end{itemize}
 
 \textbf{Neurobiology}\\
 Dr. Lee M. Miller \hfill UC Davis Department of Neurobiology, Physiology, and Behavior, Davis CA\\
 {\sl Teaching Assistant} \hfill April 2015-June 2015
 \begin{itemize}\itemsep -2pt
 \item Planned nine weeks of discussion sections with two co-TAs. Prepared material for an hour of homework review, practice problems, and discussion of lecture material and readings.
 \item Independently led 3 one hour discussion sections for a total of 75 students each week.
 %\item Held weekly office hours attended by around six students each session.
 %\item Answered numerous e-mails and arranged individual meetings for students who needed extra help.
 %\item Developed one homework assignment.
 \item Graded two short answer midterms and one short answer final exam for 200 students  \end{itemize}
 
 \textbf{Laboratory on Genes and Behavior}\\
 Dr. Rebecca D. Burwell \hfill Brown University Department of Psychology, Providence RI\\
 {\sl Teaching Assistant} \hfill January 2010-May 2010
\begin{itemize}\itemsep -2pt
\item Set up equipment for behavioral experiments run on knockout mice, including the Morris Water Maze, tail suspension and basic habituation tasks
\item Explained procedures to students and helped them run the tasks
\end{itemize}

\section{PUBLICATIONS}
%\begin{itemize} \itemsep -2pt
%\item 
{\sl Submitted}: \textbf{Popa AM}, Cruz J, Wong L, Harvey D, Angkustsiri K, Leckliter I, Perez-Edgar K, Simon TJ. Seeing Eye to Eye with Threat: Atypical Threat Bias Responses in Children with 22q11.2 Deletion Syndrome.

{\sl Submitted}: McCabe KL, \textbf{Popa AM}, Durdle C, Amato M, Cabaral M, Wong L, Harvey D, Simon TJ. Quantifying the resolution of spatial and temporal representation in children with 22q11.2 deletion syndrome.

{\sl Submitted}: Wilson JD, Baybay M, Sankar R, \textbf{Popa AM}, Stillman P. A Fast Multilayer Network Embedding Algorithm for Analyzing Group fMRI.

{\sl Submitted}: Sharma A, Singh S, Wright B, Perry A, Woodbridge DM, \textbf{Popa AM}. Scalable Motor Movement Recognition from Electroencephalography using Machine Learning. 

{\sl Submitted}: Agrawal P, Bhargavi D, Krishna G, Han X, Tevathia N, \textbf{Popa AM}, Ross N, Woodbridge DM, Zimmerman-Bier B, Bosl WJ. A Scalable Automated Diagnosis Feature Extraction System for EEGs. 

{\sl In Progress}: \textbf{Popa AM}, McCabe KL, Morgan H, Garner J, Harvey D, Amato M,  Simon TJ. Children with 22q11.2 Deletion Syndrome show Visuospatial Impairments on Bisection Tasks.

\section{PRESENTATIONS AND POSTERS} 

\begin{itemize} \itemsep -2pt
\item [] {\sl Selected from 14 Posters (5 first author, 1 last author) and 5 Presentations (4 first author) at International Conferences}\\
\item Sharma A, Singh S, Wright B, Perry A, Woodbridge DM, \textbf{Popa AM}. Scalable Motor Movement Recognition from Electroencephalography using Machine Learning. Poster Accepted at the 2nd Data Institute Conference 2019, University of San Francisco, San Francisco, CA.
\item \textbf{Popa AM}, Mayo D, Durdle C, Morgan H, Shapiro H, Ferrer E, Niendam T, Luck S, Carter C, Simon TJ. Attention and Inhibition Deficits in Youth with 22q11.2DS are  Associated with  Symptoms of Psychosis Proneness (an IBBC abstract). Poster Accepted at the 17th International Congress of the European Society for Child and Adolescent Psychiatry 2017, Geneva, Switzerland.
\item \textbf{Popa AM}, Durdle C, Morgan H, Shapiro H, Niendam T, Carter C, Luck S, Simon TJ. Highly Psychosis-Prone Adolescents show Increased Capture by Distractor Stimuli and More Effort to Inhibit Emotional Stimuli than Typically Developing Controls. Oral Accepted at the 16th International Congress on Schizophrenia Research 2017, San Diego, CA.
\item \textbf{Popa AM}, Shapiro H, Harvey D, Amato M, Cruz J, Cung N, Reyes D, Simon TJ. Children with 22q11.2 Deletion Syndrome Show Lower Spatial and Temporal Acuity Than TD Children In Continuously Varying Tasks. Abstract Accepted at the 10th Biennial International 22q11.2 Conference 2016, Sirmione, Italy.
\item \textbf{Popa AM}, Hunsaker N, Deng M, Garner J, Cruz J, Cung N, Reyes D, Simon TJ. Cortical Tissue Volumes Correlate to Cavum Septum Pellucidum Size in Children with 22q11.2 Deletion Syndrome and Typical Controls. Oral Accepted at the 71st Annual Meeting of the Society of Biological Psychiatry 2016, Atlanta, GA.
\item \textbf{Popa AM}, Beaton E, Cruz J, Wong L, Cung N, Harvey D, Simon TJ. Adaptation to a Mild Stressor in Initially Anxious Children was related to their Attention to Perceived Threat in a Dot Probe Experiment. Poster Presented at the 70th Annual Meeting of the Society of Biological Psychiatry 2015, Toronto, ON, Canada.
\end{itemize}

 \section{LEADERSHIP EXPERIENCE}
 
 \textbf{Neuroscience Initiative to Enhance Diversity} \hfill
Davis, CA\\
{\sl Student Organizer} \hfill Event held April 2016
\begin{itemize}\itemsep -2pt
\item Organized event that recruits 25 undergraduates from historically black and Hispanic serving colleges to increase representation in PhD programs
\item Coordinated 10 workshops led by faculty and students
\item Led a student data blitz and student panel on preparing your best application
\item Conducted mock interviews with the students
\end{itemize}

\textbf{Explorations, UC Davis Undergraduate Research Journal}\hfill Davis, CA\\{\sl Editor} \hfill February 2015 - August 2015\\
{\sl Managing Editor, Physical and Life Sciences} \hfill September 2015 - June 2017
\begin{itemize}\itemsep -2pt
\item Found appropriate faculty reviewers for student submitted publications
\item Communicated faculty reviewer comments to student authors
\item Assessed the reviewed and edited papers for acceptance in a yearly UC Davis publication
\item Copy-edited and formatted student articles for publication
%\item Formatted articles for publication
\end{itemize}

%\textbf{Neuroimaging Journal Club}\hfill Davis, CA\\{\sl Student Co-Facilitator} \hfill September 2013 - December 2016
%\begin{itemize}
%\item Invited faculty guests, selected papers, and led student discussions for the student-run neuroimaging journal club
%\end{itemize}

\section{COMMUNITY SERVICE}
\textbf{Intertational Rescue Committee}\hfill
Sacramento, CA\\
{\sl Refugee Empowerment Volunteer} \hfill January 2017 - October 2017
\begin{itemize}
\item Assisted recent refugees to the United States in preparing resumes and job applications
\item Helped organize the first cohort of Women's Empowerment Programming, including survey design, focus group summary, and working with the women in the program
\item Organized paperwork including financial forms to ensure compliance %with standards for non-profits
\end{itemize}

\textbf{Women in Science and Engineering}\hfill
Davis, CA\\
{\sl Mentor, Workshop Organizer, and Presenter} \hfill September 2015 - June 2017
\begin{itemize}
\item Mentored an undergraduate woman interested in pursuing an engineering graduate degree
\item Designed and presented a workshop on graduate school to 20 undergraduate women
\end{itemize}

\textbf{The Graduate Academic Achievement and Advocacy Program}\hfill Davis, CA\\{\sl Graduate Student Volunteer and Mentor} \hfill September 2013 - June 2015
\begin{itemize}
\item Advised and assisted students from underrepresented minorities on topics including graduate school applications and science writing
\item Planned and ran workshops for students from underrepresented minorities on topics including graduate school applications, research assistantships, and gap years
\item Mentored a student
\end{itemize}

\textbf{Graduate Student Assembly}\hfill Davis, CA\\{\sl Departmental Representative} \hfill September 2013 - June 2015
\begin{itemize}
\item Attended meetings and voted on policy affecting graduate students at UC Davis and in the wider UC graduate student assembly
\item Publicized information to graduate group
\end{itemize}
 
% \textbf{Brain Awareness Week}\hfill Davis, CA\\{\sl Graduate Student Presenter} \hfill March 2013 - Present
%\begin{itemize}
%\item Developed and presented a new station on Vestibular Nystagmus for a school for non-traditional high schoolers in Sacramento
%\item Presented posters at a booth at the Farmer's Market
%\item Taught neuroanatomy using sheep brains to high schoolers in Davis
%\end{itemize}

\section{AWARDS AND CERTIFICATES}
UC Davis Graduate Student Asssembly Travel Award \hfill 2015-16 Academic Year\\
UC Davis FUTURE Certificate Track \hfill 2015-16 Academic Year\\
UC Davis Graduate Student Asssembly Travel Award \hfill 2014-15 Academic Year\\
ERP Boot Camp (Dr. Steven J. Luck) \hfill Completed July 2014\\
UC Davis Graduate Student Asssembly Travel Award \hfill 2013-14 Academic Year

 \section{TECHNICAL SKILLS} 
 \begin{itemize}[leftmargin=-2pt] \itemsep -2pt
\item [] Software including: Python, R, Matlab, Jupyter, git, SPSS, DataGraph, \LaTeX\ 
\item []Packages including: sklearn, pandas, numpy, scipy, matplotlib, TensorFlow, keras, ggplot, nlme
 \end{itemize}


\end{resume}

\end{document}







