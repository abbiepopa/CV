% LaTeX resume using res.cls
\documentclass[line,margin,10pt]{res} 
\usepackage{enumitem}
\usepackage{hyperref}
\oddsidemargin -0.3in
\textwidth 5.75in

\begin{document}

\name{Abbie M. Popa, Ph.D.}
 
\begin{resume}
\moveleft 0.5\hoffset\centerline 
{\hyperref[abbiepopa@gmail.com]{abbiepopa@gmail.com} \quad \quad \quad \quad  \quad \quad \quad \quad \quad \quad \quad \quad \quad \quad \quad \quad \quad \quad \quad \quad \quad\quad \quad \quad \quad \quad \quad \quad \quad \quad \quad \quad  \quad \quad \quad   401-440-5228}
\moveleft 0.5\hoffset\centerline 
{\hspace{0.05cm} \hyperref[linkedin.com/in/abbiepopa]{linkedin.com/in/abbiepopa} \quad \quad \quad \quad \quad \quad \quad \quad \quad \quad \quad\quad \quad \quad \quad \quad \quad \quad \quad \quad \quad \quad  \quad \quad \quad  \quad \quad \quad   \quad \hyperref[github.com/abbiepopa]{github.com/abbiepopa}}
 
\section{EDUCATION} 
\textbf{Ph.D.}, University of California at Davis, Neuroscience \hfill June 2018\\
%Dissertation: Behavior and EEG Testing of Adolescent Anxiety\\
%Data Science Initiative Affiliation\\
%\emph{Thesis Topic}: Anxiety Impacts Attentional and Inhibitory Control in Adolescence\\
\textbf{Honors Sc.B.}, Brown University, Cognitive Neuroscience \hfill May 2011\\
                % \sl will be bold italic in New Century Schoolbook (or
	        % any postscript font) and just slanted in
		% Computer Modern (default) font
%                Major: Cognitive Neuroscience

 \section{TECHNICAL SKILLS} 
 \begin{itemize}[leftmargin=-2pt] \itemsep -2pt
\item [] Software including: Python, R, Jupyter, git%, SQL
\item []Packages including: sklearn, pandas, numpy, scipy, matplotlib, ggplot, fastai, lm, lme, dplyr%, stringr, tidyr, nlme, pytorch, 
 \end{itemize}

\section{TECHNICAL EXPERIENCE}
\textbf{The Data Institute at the University of San Francisco} \hfill San Francisco, CA \\
{\sl Data Science Postdoctoral Fellow} \hfill August 2018 - present
\begin{itemize} \itemsep -2pt
\item Using self-supervised learning techniques including feature embeddings to better describe nodes in brain networks from patients with schizophrenia and healthy controls
\item Using machine learning techniques including random forests and support vector machines to classify non-linear features in EEG from infants who were born preterm
\item Contributing to a collaborative reading and practice group on Reinforcement Learning
\item Consulting on collaborations with 3 private sector companies
\item Instructed and developed an Introduction to Data Science class for 36 undergraduates
\end{itemize}

\textbf{UC Davis MIND Institute} \hfill Sacramento, CA \\
{\sl Ph.D. Researcher} \hfill September 2012 - June 2018
\begin{itemize} \itemsep -2pt
%\item Completed 8 posters for scientific conferences
\item Used k-means clustering to organize behavioral, eye-tracking, and self-report measures
%\item Used ICA to isolate brain activity from noise in EEG data
\item Used mixed effects linear modeling to identify patterns in participants' behavior over time
%\item Predicted children's outcomes from regression on 500GB of MRI brain development data
\item Developed 6 child-friendly computerized behavior tests (disguised as games)
%\item Programmed data analyses and visual stimuli using R, Python, and Matlab
\item Trained and mentored 6 junior research assistants and 7 volunteer interns
%\item Teaching assistant for Neurobiology (class of 200 students) responsible for leading 3 1-hour long discussion sections each week
\item Resulted in 4 manuscripts (in progress) and 18 conference presentations (completed)
%\item Received 3 travel awards in recognition of my work
\end{itemize}

\textbf{Davis Incubator Group} \hfill Student organized group at UC Davis\\
%Davis, CA \hfill January 2016 - Present\\
{\sl President} \hfill September 2016 - June 2018\\
{\sl Member} \hfill January - August 2016 
\begin{itemize} \itemsep -2pt
%\item Completed online coursework in python, SQL, and machine learning
\item Completed 2 collaborative Kaggle image classification challenges using skimage, TensorFlow, tflearn, keras, OpenCV and PIL on an AWS machine to efficiently localize and classify images through convolutional neural networks for datasets up to 100 GB.
\item Completed a collaborative Driven Data competition using pandas and sklearn to finish in the top 10\% of competitors
%\item Developed individual projects using tools including sklearn in Python, and nnet in R
%\item Organized and scheduled meetings for a group of 6-8 data scientists %to practiced coding, machine learning, and share data science skills
\end{itemize}

\textbf{UC Davis Data Science Initiative} \hfill  Davis, CA \\
{\sl Affiliate} \hfill April 2016 - June 2018
\begin{itemize} \itemsep -2pt
\item Contributed to collaborative reading and practice groups on {\sl Think Python} and {\sl An Introduction to Statistical Learning} 
\item Consulted with members of the UC Davis community on data science problems from twitter scraping to genomics as part of team seminars
\end{itemize}
 
\section{LEADERSHIP AND COMMUNITY EXPERIENCE}
\begin{itemize}\itemsep -2pt

\item [] \textbf{Software Carpentry} \\ {\sl Certified Instructor for bash, git, R, and Python} \hfill May 2018 - Present

\item [] \textbf{International Rescue Committee} \\ {\sl Refugee Empowerment Volunteer} \hfill January 2017 - October 2017

\item [] \textbf{Explorations, UC Davis Undergraduate Research Journal} \\
{\sl Editor} \hfill February 2015 - August 2015\\
{\sl Managing Editor, Physical and Life Sciences} \hfill September 2015 - June 2017
  
%\item  [] \textbf{Neuroscience Initiative to Enhance Diversity} \\ {\sl Student Organizer} \hfill Event held April 2016

%\item [] \textbf{Student Recruitment and Retention Center} \\ {\sl Mentor, Workshop Organizer, and Presenter} \hfill September 2013 - June 2017

%\item [] \textbf{Graduate Student Assembly} \\ {\sl Departmental Representative} \hfill September 2013 - June 2015

\end{itemize}
 

\end{resume}

\end{document}







