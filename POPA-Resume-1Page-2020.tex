% LaTeX resume using res.cls
\documentclass[line,margin,10pt]{res} 
\usepackage{enumitem}
\usepackage{hyperref}
\oddsidemargin -0.3in
\textwidth 5.75in

\begin{document}

\name{Abbie M. Popa, Ph.D.}
 
\begin{resume}
\moveleft 0.5\hoffset\centerline 
{\hyperref[abbiepopa@gmail.com]{abbiepopa@gmail.com} \quad \quad \quad \quad  \quad \quad \quad \quad \quad \quad \quad \quad \quad \quad \quad \quad \quad \quad \quad \quad \quad\quad \quad \quad \quad \quad \quad \quad \quad \quad \quad \quad  \quad \quad \quad   401-440-5228}
\moveleft 0.5\hoffset\centerline 
{\hspace{0.05cm} \hyperref[linkedin.com/in/abbiepopa]{linkedin.com/in/abbiepopa} \quad \quad \quad \quad \quad \quad \quad \quad \quad \quad \quad\quad \quad \quad \quad \quad \quad \quad \quad \quad \quad \quad  \quad \quad \quad  \quad \quad \quad   \quad \hyperref[github.com/abbiepopa]{github.com/abbiepopa}}
 


\section{TECHNICAL EXPERIENCE}
\textbf{App Annie} \hfill San Francisco, CA\\
{\sl Senior Data Scientist} \hfill August 2019 - present
\begin{itemize} \itemsep -2pt
\item Building machine learning models to help business customers navigate the mobile app marketplace
\end{itemize}

\textbf{Cogitativo} (healthcare consulting start-up) \hfill Berkeley, CA\\
{\sl Data Science Intern} \hfill April 2019 - June 2019
\begin{itemize} \itemsep -2pt
%\item Identified instances of non-recommended healthcare procedures from millions of records stored in AWS Redshift
%\item Developed tables of metrics to summarize new payment integrity findings for our clients
%\item Built a Tableau dashboard to visualize anomalous behavior of home healthcare providers
\item Developed models, metrics, and Tableau dashboards from millions of records stored in AWS Redshift to understand healthcare operations data.
\end{itemize}

\textbf{The Data Institute at the University of San Francisco} \hfill San Francisco, CA \\
{\sl Data Science Postdoctoral Fellow} \hfill August 2018 - July 2019
\begin{itemize} \itemsep -2pt
\item Described nodes in brain networks from patients with schizophrenia and healthy controls using self-supervised machine learning (feature embeddings)
\item Contributing to a web-based app for mass processing of electroencephalography (EEG) files in the cloud
%\item Classifying nonlinear features in EEG from infants who were born preterm using machine learning techniques including random forests and support vector machines
\item Consulted on collaborations with 3 private sector companies using computer vision and natural language processing
\item Managed projects resulting in 2 published IEEE proceedings using PySpark and SparkML to make predictions on EEG data in collaboration with other researchers
%\item Contributed to a collaborative reading and practice group on Reinforcement Learning 
\item Instructed and developed an Introduction to Data Science class for 36 undergraduates
\item Completed deep learning coursework using fast.ai and PyTorch (fast.ai certification course)
\end{itemize}

\textbf{University of California at Davis} \hfill Davis, CA \\
{\sl Ph.D. Researcher} \hfill September 2012 - June 2018
\begin{itemize} \itemsep -2pt
\item Used techniques including mixed effects linear modeling, k-means clustering, and ICA interpret data in participants' behavior and brain activity over time
%\item Used k-means clustering to organize behavioral, eye-tracking, and self-report measures
%\item Used ICA to isolate brain activity from noise in EEG data
%\item Developed 6 child-friendly computerized behavior tests (disguised as games)
\item Trained and mentored 6 junior research assistants and 7 volunteer interns
%\item Teaching assistant for Neurobiology (class of 200 students) responsible for leading 3 1-hour long discussion sections each week
\item Produced 4 manuscripts (2 accepted) and completed 18 conference presentations
\item Helped organize and run Davis Incubator Group, a group for graduate students interested in machine learning and data science
\item Consulted with members of the UC Davis community on data science problems from twitter scraping to genomics as an affiliate of the Data Science Initiative
\end{itemize}


 \section{TECHNICAL SKILLS} 
 \begin{itemize}[leftmargin=-2pt] \itemsep -2pt
\item [] Software include: PySpark, Python, R, git, PostgreSQL
\item []Packages include: sklearn (contributor), pandas (contributor), NumPy, SparkML, PySparkSQL%, SciPy, matplotlib, ggplot, lm, lme, dplyr, fastai, stringr, tidyr, nlme, pytorch, 
 \end{itemize}


\section{EDUCATION} 
\textbf{Ph.D.}, University of California at Davis, Neuroscience\\
%Dissertation: Behavior and EEG Testing of Adolescent Anxiety\\
%Data Science Initiative Affiliation\\
%\emph{Thesis Topic}: Anxiety Impacts Attentional and Inhibitory Control in Adolescence\\
\textbf{Honors Sc.B.}, Brown University, Cognitive Neuroscience\\
                % \sl will be bold italic in New Century Schoolbook (or
	        % any postscript font) and just slanted in
		% Computer Modern (default) font
%                Major: Cognitive Neuroscience
 
\section{LEADERSHIP AND COMMUNITY EXPERIENCE}
\begin{itemize}\itemsep -2pt

\item [] \textbf{Software Carpentry} \\ {\sl Certified Instructor for bash, git, R, and Python} \hfill May 2018 - Present

\item [] \textbf{International Rescue Committee} \\ {\sl Refugee Empowerment Volunteer} \hfill January 2017 - October 2017

\item [] \textbf{Explorations, UC Davis Undergraduate Research Journal} \\
{\sl Editor} \hfill February 2015 - August 2015\\
{\sl Managing Editor, Physical and Life Sciences} \hfill September 2015 - June 2017
  
%\item  [] \textbf{Neuroscience Initiative to Enhance Diversity} \\ {\sl Student Organizer} \hfill Event held April 2016

%\item [] \textbf{Student Recruitment and Retention Center} \\ {\sl Mentor, Workshop Organizer, and Presenter} \hfill September 2013 - June 2017

%\item [] \textbf{Graduate Student Assembly} \\ {\sl Departmental Representative} \hfill September 2013 - June 2015

\end{itemize}
 

\end{resume}

\end{document}







