% LaTeX resume using res.cls
\documentclass[line,margin,10pt]{res} 
\usepackage{enumitem}
\usepackage{hyperref}
\oddsidemargin -0.3in
%\topmargin -0.3in
\textwidth 5.75in

\hypersetup{
    colorlinks=true,
    linkcolor=blue,
    filecolor=magenta,      
    urlcolor=cyan,
}

\urlstyle{same}

\begin{document}

\name{Abbie M. Popa, Ph.D.}
 
\begin{resume}
\moveleft 0.5\hoffset\centerline 
{\hyperref[abbiepopa@gmail.com]{abbiepopa@gmail.com} \quad \quad \quad \quad  \quad \quad \quad \quad \quad \quad \quad \quad \quad \quad \quad \quad \quad \quad \quad \quad \quad\quad \quad \quad \quad \quad \quad \quad \quad \quad \quad \quad  \quad \quad \quad   401-440-5228}
\moveleft 0.5\hoffset\centerline 
{\hspace{0.05cm} \hyperref[linkedin.com/in/abbiepopa]{linkedin.com/in/abbiepopa} \quad \quad \quad \quad \quad \quad \quad \quad \quad \quad \quad\quad \quad \quad \quad \quad \quad \quad \quad \quad \quad \quad  \quad \quad \quad  \quad \quad \quad   \quad \hyperref[github.com/abbiepopa]{github.com/abbiepopa}}
 
{\vspace{-0.25cm}}

\section{TECHNICAL EXPERIENCE}
\textbf{Merck} \hfill Remote\\
{\sl Associate Director of Data Science} \hfill September 2021 - Present
\begin{itemize} \itemsep -2pt
	\item Delivered informed patient file for triple-negative breast cancer market-sizing for Keytruda, Merck's premier immunotherapy drug
	\item Produced drug regimen summary and forecasting for renal cell carcinoma to support the launch of Merck's first CTLA4 inhibitor (immunotherapy)
	\item Building forecasting tools for self-service estimation and experimentation
\end{itemize}
{\vspace{-0.25cm}}
\textbf{App Annie} \hfill San Francisco, CA\\
{\sl Staff  Data Scientist} \hfill April 2021 - August 2021
\begin{itemize} \itemsep -2pt
\item Built machine learning models to help businesses navigate the mobile app marketplace
\item Led two data scientists on prioritization and planning of model improvements
\item Sustained and improved downloads and revenue models, App Annie's core product
\item Modeled user acquisition for apps across four paid and organic \href{https://www.appannie.com/en/insights/product-announcements/understand-paid-vs-organic-downloads-with-improved-download-channel-report/}{channels}
\item Received quarterly company-wide \href{https://www.linkedin.com/pulse/meet-abbie-popa-staff-data-scientist-from-our-amer-region-thomas/?trackingId=BPJh%2B6dFR%2Baz%2BUpkpGznRA%3D%3D}{award} for collaboration and going above and beyond
\item Patent: Stolorz, Paul; Yatabe Rodriguez, Tada; Popa, Abbie. 2022. Fast Estimation of Downloads for Apps at Launch. Patent Number 20220311834.
\end{itemize}
{\vspace{-0.25cm}}
{\sl Senior Data Scientist} \hfill August 2019 - April 2021
\begin{itemize} \itemsep -2pt
\item Modeled likely causes of anomalous events in the Insights (formerly \href{https://www.appannie.com/en/insights/product-announcements/data-stories-labs/}{Data Stories}) product
\item Increased the accuracy of our core model in predicting extremely well-performing apps, resulting in a significant decrease in customer support tickets
%\item Led adoption of best practices including version control, continuous integration, and code review among the data science team
\end{itemize}
{\vspace{-0.25cm}}
\textbf{Cogitativo} (healthcare consulting start-up) \hfill Berkeley, CA\\
{\sl Data Science Intern} \hfill April 2019 - June 2019
\begin{itemize} \itemsep -2pt
\item Identified non-recommended procedures from millions of records in AWS Redshift
\item Developed tables of metrics to summarize new payment integrity findings for our clients
\item Built a Tableau dashboard to visualize anomalous behavior of home healthcare providers
%\item Developed models, metrics, and Tableau dashboards from millions of records stored in AWS Redshift to understand healthcare operations data.
\end{itemize}
{\vspace{-0.25cm}}
\textbf{The Data Institute at the University of San Francisco} \hfill San Francisco, CA \\
{\sl Data Science Postdoctoral Fellow} \hfill August 2018 - July 2019
\begin{itemize} \itemsep -2pt
\item Described \href{https://www.youtube.com/watch?v=8Ng_aMgIZLw&feature=youtu.be}{nodes in brain networks} from patients with schizophrenia and healthy controls using self-supervised machine learning (feature embeddings) resulting in one publication in Network Science (respected peer-reviewed journal)
%\item Contributing to a web-based app for mass processing of electroencephalography (EEG)
%\item Classifying nonlinear features in EEG from infants who were born preterm using machine learning techniques including random forests and support vector machines
\item Consulted on collaborations with 3 private sector companies using computer vision and natural language processing including deep learning
\item Managed projects resulting in 2 published IEEE proceedings using PySpark and SparkML to make predictions on EEG data in collaboration with other researchers
%\item Contributed to a collaborative reading and practice group on Reinforcement Learning 
%\item Instructed and developed an Introduction to Data Science class for 36 undergraduates
%\item Completed deep learning coursework using fast.ai and PyTorch (fast.ai certification course)
\end{itemize}
{\vspace{-0.25cm}}
\textbf{University of California at Davis} \hfill Davis, CA \\
{\sl Ph.D. Researcher} \hfill September 2012 - June 2018
\begin{itemize} \itemsep -2pt
\item Used mixed effects linear modeling, k-means clustering, and ICA to interpret participants' behavior and brain activity resulting in 4 publications and 18 conference presentations
%\item Used k-means clustering to organize behavioral, eye-tracking, and self-report measures
%\item Used ICA to isolate brain activity from noise in EEG data
%\item Developed 6 child-friendly computerized behavior tests (disguised as games)
\item Trained and mentored 6 junior research assistants and 7 volunteer interns
%\item Teaching assistant for Neurobiology (class of 200 students) responsible for leading 3 1-hour long discussion sections each week
\item Consulted with members of the UC Davis community on data science problems from twitter scraping to genomics as an affiliate of the Data Science Initiative
\end{itemize}
{\vspace{-0.25cm}}


 \section{TECHNICAL SKILLS} 
 \begin{itemize}[leftmargin=-2pt] \itemsep -2pt
\item [] Software including: PySpark, Python, R, git, PostgreSQL, Tableau
\item []Packages including: PySparkSQL, SparkML, sklearn (contributor), pandas (contributor), NumPy%, plotnine, SciPy, matplotlib, ggplot, lm, lme, dplyr, fastai, stringr, tidyr, nlme, pytorch, 
 \end{itemize}
{\vspace{-0.25cm}}

\section{EDUCATION} 
\textbf{Ph.D.}, University of California at Davis, Neuroscience\\
%Dissertation: Behavior and EEG Testing of Adolescent Anxiety\\
%Data Science Initiative Affiliation\\
%\emph{Thesis Topic}: Anxiety Impacts Attentional and Inhibitory Control in Adolescence\\
\textbf{Honors Sc.B.}, Brown University, Cognitive Neuroscience%\\
                % \sl will be bold italic in New Century Schoolbook (or
	        % any postscript font) and just slanted in
		% Computer Modern (default) font
%                Major: Cognitive Neuroscience
 
%\section{LEADERSHIP AND COMMUNITY EXPERIENCE}
%\begin{itemize}\itemsep -2pt
%
%\item [] \textbf{Software Carpentry} \\ {\sl Certified Instructor for bash, git, R, and Python} \hfill May 2018 - Present
%
%\item [] \textbf{International Rescue Committee} \\ {\sl Refugee Empowerment Volunteer} \hfill January 2017 - October 2017
%
%\item [] \textbf{Explorations, UC Davis Undergraduate Research Journal} \\
%{\sl Editor} \hfill February 2015 - August 2015\\
%{\sl Managing Editor, Physical and Life Sciences} \hfill September 2015 - June 2017
  
%\item  [] \textbf{Neuroscience Initiative to Enhance Diversity} \\ {\sl Student Organizer} \hfill Event held April 2016

%\item [] \textbf{Student Recruitment and Retention Center} \\ {\sl Mentor, Workshop Organizer, and Presenter} \hfill September 2013 - June 2017

%\item [] \textbf{Graduate Student Assembly} \\ {\sl Departmental Representative} \hfill September 2013 - June 2015

%\end{itemize}
 

\end{resume}

\end{document}







