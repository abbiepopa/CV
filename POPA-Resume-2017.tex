% LaTeX resume using res.cls
\documentclass[line,margin,10pt]{res} 
%\usepackage{helvetica} % uses helvetica postscript font (download helvetica.sty)
%\usepackage{newcent}   % uses new century schoolbook postscript font 
\usepackage{enumitem}
\usepackage{hyperref}
\oddsidemargin -0.25in
\textwidth 5.75in

\begin{document}


\name{Abbie M. Popa, ScB \quad \quad \quad \quad \quad \quad \quad \quad \quad \quad \quad UC Davis Neuroscience Graduate Group}
% \address used twice to have two lines of address
%\address{1333 Arlington Blvd Apt 6, Davis, CA 95616}
%\address{phone:(401) 440 5228 e-mail:abbiepopa@gmail.com}
%\address{linkedin.com/in/abbiepopa}
%\address{github.com/abbiepopa}
 
\begin{resume}
\moveleft 0.5\hoffset\centerline 
{\hyperref[abbiepopa@gmail.com]{abbiepopa@gmail.com} \quad \quad \quad \quad  \quad \quad \quad \quad \quad \quad \quad \quad \quad \quad \quad \quad \quad \quad \quad \quad \quad\quad \quad \quad \quad \quad \quad \quad \quad \quad \quad \quad  \quad \quad \quad   401-440-5228}
\moveleft 0.5\hoffset\centerline 
{\hspace{0.05cm} \hyperref[linkedin.com/in/abbiepopa]{linkedin.com/in/abbiepopa} \quad \quad \quad \quad \quad \quad \quad \quad \quad \quad \quad\quad \quad \quad \quad \quad \quad \quad \quad \quad \quad \quad  \quad \quad \quad  \quad \quad \quad   \quad \hyperref[github.com/abbiepopa]{github.com/abbiepopa}}
 
\section{EDUCATION} 
\textbf{PhD University of California at Davis} \hfill (September 2012 - December 2017)\\
Major: Cognitive Neuroscience with a Data Science Initiative Affiliation\\
%\emph{Thesis Topic}: Anxiety Impacts Attentional and Inhibitory Control in Adolescence\\
\textbf{Honors ScB.}, Brown University \hfill (September 2006-December 2010)\\
                % \sl will be bold italic in New Century Schoolbook (or
	        % any postscript font) and just slanted in
		% Computer Modern (default) font
                Major: Cognitive Neuroscience
                
 \section{TECHNICAL SKILLS} 
 \begin{itemize}[leftmargin=-2pt] \itemsep -2pt
\item [] Python, R, Matlab, Jupyter, git, SQL, SPSS, DataGraph, \LaTeX\ , HTML, CSS
\item []Packages including TensorFlow, sklearn, ggplot, nlme, pandas,  tflearn
%SPSS
 %\item[]Experienced: Scheme, OCaml, Java, Jupyter
 %\item[] If I have a computer and an internet connection, I can learn it!
 \end{itemize}
		%							\emph{Thesis Title}: Concordance of Movement and Heart Rate Responses in Fetuses at Risk for Autism\\
								
%\section{AWARDS AND CERTIFICATES}
%UC Davis Graduate Student Asssembly Travel Award \hfill 2014-15 Academic Year\\
%ERP Boot Camp (Dr. Steven J. Luck) \hfill Completed July 2014\\
%UC Davis Graduate Student Asssembly Travel Award \hfill 2013-14 Academic Year

%\section{AFFILIATIONS}
%Society of Biological Society, \textit{Student Member} \hfill 2013-present\\
%Society for Neuroscience, \textit{Student Member} \hfill 2013-present\\
%Association for Women in Science, \textit{Student Member} \hfill 2013-present\\
%American Association for the Advancement of Science, \textit{Student Member} \hfill 2013-2014
                
%\section {SCORES \emph{\&} \\ GPA}
%\begin{tabular}{@{}l}                 
%Overall GPA 8 Semesters: 3.67
%\\ Concentration GPA 8 Semesters: 3.70
%\\ Overall GPA Last 4 Semesters: 4.00
%\end{tabular} \hfill
%\begin{tabular}{l@{\hspace*{1em}}}
 %GRE Verbal: 690
%\\ GRE Quantitative: 800
%\\ GRE Writing: 6
%\end{tabular}

%\section{DISSERTATION}
%\textbf{Anxiety Impacts Attentional and Inhibitory Control in Adolescence: An ERP Study}
%Mentor: Dr. Tony J. Simon \hfill June 2013 - present\\
%A study of 40 adolescents with generalized anxiety disorder and 20 typical controls examining attentional and inhibitory control using event-related potentials, heart rate variability, and behavioral data. Participants will complete four tasks as well as self and parent reports of daily function.


%\section{PUBLICATIONS}
%\begin{itemize} \itemsep -2pt
%\item 
%{\sl In Progress}: \textbf{Popa AM}, Cruz J, Wong L, Harvey D, Angkustsiri K, Leckliter I, Perez-Edgar K, Simon TJ. Anxious Children with 22q11.2 Deletion Syndrome show Atypical Adaptation Responses to Mild Threat Stimuli.

%\section{PRESENTATIONS AND POSTERS} 
%\begin{itemize} \itemsep -2pt
%\item \textbf{Popa AM}, Beaton E, Cruz J, Wong L, Cung N, Harvey D, Simon TJ. Adaptation to a Mild Stressor in Initially Anxious Children was related to their Attention to Perceived Threat in a Dot Probe Experiment. Poster Presented at the 70th Annual Meeting of the Society of Biological Psychiatry 2015, Toronto, ON, Canada.
%\item \textbf{Popa AM}, Angkustsiri K, Brahmbhatt K, Cruz J, Cung N, Harvey D, Leckliter I, Reyes D, Shapiro H, Wong L, Simon TJ. Impact of Cognitive-Affective Interactions on Attentional Control, Inhibition, and Temporal Attention. Oral Presentation Accepted at the 21st International Scientific Meeting of the Velo-Cardio-Facial Syndrome Educational Foundation, Inc 2014, Las Vegas Nevada.
%\item \textbf{Popa AM}, Angkustsiri K, Brahmbhatt K, Cruz J, Cung N, Leckliter I, Quintero A, Reyes D, Shapiro H, Simon TJ. Timecourse of Response to Threat Stimuli in Children with 22q11.2 Deletion Syndrome Informs Understanding of Anxiety. Oral Presentation Accepted at the 9th Annual Meeting of the International 22q11.2 Foundation 2014, Mallorca, Spain.
%\item Angkustsiri K, \textbf{Popa AM}, Simon TJ. Atypical Pupillary Responses To Emotional Faces in Children With Chromosome 22q11.2 Deletion Syndrome. Oral Presentaion Accepted at the Pediatric Academic Societies and Asian Society for Pediatric Research Joint Meeting 2014 Vancouver, BC, Canada.
%\item \textbf{Popa AM}, Angkustsiri K, Brahmbhatt K, Cruz J, Cung N, Leckliter I, Quintero A, Reyes D, Shapiro H, Simon TJ. Atypical Adaptation  Responses to Threat Stimuli in Children with Chromosome 22q11.2 Deletion Syndrome. Poster Presented at the 69th Annual Meeting of the Society of Biological Psychiatry 2014, New York, NY.
%\item Cruz JR, \textbf{Popa AM}, Wong LM, Angkustsiri K, Shapiro H, Fox N, Pine D, Perez-Edgar K, Simon TJ. Examining Attention Bias Towards Threat to Understand Cognition and Anxiety Interactions in 22q11.2DS. Poster presented at the 20th International Scientific Meeting of the Velo-Cardio-Facial Syndrome Educational Foundation 2013, Dublin, Ireland.
%\item Cruz JR, Wong LM, Angkustsiri K, \textbf{Popa AM}, Shapiro HM, Fox N, Pine D, Perez-Edgar K, Simon TJ. Children with Chromosome 22q11.2 Deletion Syndrome Exhibit High Levels of Anxiety and Threat Bias in a Dot Probe Experiment. Poster presented at the 68th Annual Meeting of the Society of Biological Psychiatry 2013, San Francisco, CA.
%\item Sullivan, M. C., Miller, R. J., Winchester, S. B., Barcelos, M., Oliveira, E., \& \textbf{Popa, A}. (2012, May). Developmental Origins Theory and HPA axis function: Evidence from a Longitudinal Study of Preterm Infants at Young Adulthood.  Poster presented at the Massachusetts General Hospital, Yvonne L. Munn Center for Nursing Research, Nursing Research Expo, Boston, MA.
%\end{itemize}

%\section{SENIOR HONORS THESIS}
%\section{PROJECTS}


%\textbf{\href{https://github.com/davisincubator/sashimdig}{Kaggle: Fish Identification}}
%\begin{itemize} \itemsep -2pt
%\item Used skimage, TensorFlow, and tflearn to classify fish images from a kaggle data set
%\item Preprocessed images using PIL and skimage and stored data in HDF5 for efficiency
%\item Collaborated with three other data scientists to apply convolutional neural networks
%\end{itemize}

%\textbf{FastText Horror Author Classification}
%\begin{itemize}\itemsep -2pt
%\item Used Facebook's publically available package FastText to classify text 
%\item Pre-processed text populated from Project Gutenburg public domain database using bash
%\item Achieved 80\% accuracy for single sentences between two horror authors 
%\end{itemize}
%
%\textbf{\href{http://davisig.org/blog/2016/07/26/KDQ_AnimalShelter}{Kaggle Done Quick: Animal Shelter Predictions}}
%\begin{itemize}\itemsep -2pt
%\item Cleaned and re-binned data for feature dimensionality reduction
%\item Performed multinomial logistic regression using the nnet package of R to predict five possible outcomes for shelter animal and generated confusion matrices to visualize data
%\item Completed project in under 4 hours
%\end{itemize}

%\textbf{\href{https://github.com/davisincubator/digblood}{Driven Data: Blood Drive Donations}}
%\begin{itemize}\itemsep -2pt
%\item Cleaned data in pandas to account for outliers and unusual distributions
%\item Used sklearn to cross-validate multiple models, selecting random forest classification
%\item Collaborated with three other data scientists to finish in the top 10\% of competitors
%\end{itemize}

\section{TECHNICAL EXPERIENCE}
\textbf{PhD Researcher} \hfill UC Davis MIND Institute and Neuroscience Graduate Group\\ 
Davis, CA \hfill September 2012 - present
\begin{itemize} \itemsep -2pt
%\item Completed 8 posters for scientific conferences
\item Developed 6 child-friendly computerized behavior tests (disguised as games)
\item Used k-means clustering to classify children as "copers" or "strugglers" based on behavioral, eye-tracking, and self-report measures
\item Used ICA to isolate brain activity from noise in EEG data
\item Used non-linear modeling to classify participants behavior over time
\item Used linear regression to correlate trajectories of brain development and children's outcomes in a large (approximately 500 GB) dataset
\item Programmed data analyses and visual stimuli using R, Python, and Matlab
\item Trained and mentored four junior research assistants and seven volunteer interns
\end{itemize}

\textbf{Davis Incubator Group} \hfill Student organized group at UC Davis\\
%Davis, CA \hfill January 2016 - Present\\
President \hfill (September 2016 - present)\\
Member \hfill (January - August 2016) 
\begin{itemize} \itemsep -2pt
\item Completed online coursework in python, SQL, and machine learning
\item Completed two collaborative Kaggle competitions using skimage, TensorFlow, tflearn, keras, OpenCV and PIL on an AWS machine to efficiently localize and classify images through convolutional neural networks for datasets up to 100 GB.
\item Completed a collaborative Driven Data competition using Pandas and sklearn to finish in the top 10\% of competitors
\item Developed and shared individual work using tools including Facebook's publically available package FastText
\item Organized and scheduled meetings for a group of 6-8 data scientists to practiced coding, machine learning, and share data science skills
\end{itemize}

%\textbf{22q11.2 Research Center and Clinic}\\
%Dr. Tony J. Simon \hfill MIND Institute at UC Davis, Davis CA\\{\sl Graduate Student Researcher} \hfill June 2013 - present\\ {\sl Rotation Student} \hfill April 2013 - June 2013
%\begin{itemize} \itemsep -2pt
%\item Analysis of behavioral, eye gaze, and pupillometric data from a dot probe threat bias experiment as they relate to self report measures of anxeity and cognition in 47 children with 22q11.2 deletion syndrome and 32 typically developing children
%\item Assistance in generating posters based on these data
%\item End of quarter rotation talk
%\item Design, pilot testing, data collection, and analysis of four ERP experiments examining adolescent's with 22q11.2 deletion syndrome and typical controls on attentional and inhibitory control using neutral and emotional stimuli for a five-year NIH funded grant.
%\item Task development for five behavioral experiments on the same study
%\item Analysis of resting state fMRI data from over a hundred participants with and without 22q11.2 deletion syndrome to examine differences in three networks isolated using ICA

%\item Preparing a manuscript regarding the previous DPTB work
%\end{itemize}	
%	\textbf{Concordance of Movement and Heart Rate Responses in Fetuses at Risk for Autism}\\
%Dr. Stephen J. Sheinkopf \hfill Brown University, Providence RI \\
%{\sl Senior Honors Thesis Project} \hfill September 2009-December 2010
%\begin{itemize}\itemsep-2pt
%\item Honors thesis for the completion of requirements associated with departmental honors
%\item Developed a research question and methodology; collected, analyzed and reported on data
%\item Based on actocardiograph data, compared a sample of fetuses at high risk for autism (one or more confirmed siblings or parents with autism) to a group of normal controls on concordance between movements and heart rate.  This was observed at rest and in response to social and asocial stimuli
%\item Coded actocardiograph strips for different heart rate and movement patterns
%\item Analyzed data using a chi-squared analysis in SPSS
%\item Prepared a paper reporting the results
%\item Defended the thesis in an open format including a presentation and questions, which the general public was invited to
 %\end{itemize}

%\section{CLINICAL \\ EXPERIENCE} \textbf{The Groden Center}\\
%Providence, RI\\
%{\sl Treatment Teacher} \hfill June 2009-December 2009 
               
   %              \begin{itemize}  \itemsep -2pt % reduce space between items
      %           \item Worked as a treatment teacher in a classroom for adolescents with severe autism and profound behavioral problems doing individualized lessons and therapies
         %       \item   Helped take children and adolescents with severe autism on community field trips                 \end{itemize}
 %\textbf{Writers' Group}\\
   %The Swearer Center at Brown University, Providence, RI\\
      %          {\sl Facilitator} \hfill            February 2009-May 2009 
              
         %        \begin{itemize}  \itemsep -2pt %reduce space between items
            %     \item Planned and facilitiated weekly lessons and activities with a student organized group that prepares writing workshops for adults with developmental disabilites in the local community.
                 %\item Planned weekly lessons and activities
                 %\item With two other students, co-facilitated a group of ten adults, explaining the activities and helping them write and share their work
               %  \end{itemize} 
                 
                  %     \textbf{Bonn Nontapum }\\
                 %Cross Cultural Solutions, Bangkok, Thailand\\
                %{\sl Volunteer} \hfill        September 2008- December 2008
                
                  %\begin{itemize}
			%						\item Performed play and life skill activities with children at a home for children with special needs in Thailand
                   %\item Took a semester off from school to volunteer at a home for special needs children in Thailand.
                   %\item Played games and helped with lunchtime in a ward for children with cerebral palsy
                   %\item Helped with crafts, sing-a-longs and other activities in a ward for children with autism
                   %\item Helped run free``playtime" for children with non-specific developmental disabilities
                   %\end{itemize} 
 
 %\section{TEACHING\\EXPERIENCE} 
 %\textbf{Teaching Assistant} \hfill UC Davis Department of Neurobiology, Physiology, and Behavior\\ 
 %Davis CA \hfill April 2015-June 2015
 %\begin{itemize}\itemsep -2pt
 %\item Planned nine weeks of discussion sections in collaboration with two co-TAs. Prepared material for an hour of homework review, practice problems, and discussion of lecture material and readings.
 %\item Independently led 3 one hour discussion sections for a total of 75 students each week and held weekly office hours.
% \item Answered numerous e-mails and arranged individual meetings for students who needed extra help.
 %\item Developed one homework assignment.
 %\item Graded three short answer exams for 200 students with one co-TA.
 %\end{itemize}
 
% \textbf{Laboratory on Genes and Behavior}\\
 %Dr. Rebecca D. Burwell \hfill Brown University Department of Psychology, Providence RI\\
 %{\sl Teaching Assistant} \hfill January 2010-May 2010
%\begin{itemize}\itemsep -2pt
%\item Responsible for setting up equipment for behavioral experiments run on knockout mice, including the Morris Water Maze, tail suspension and basic habituation tasks
%\item Explained procedures to students and helped them run the tasks
%\item Cleaned up after the lab
%\end{itemize}
                    
 \section{LEADERSHIP AND\\ COMMUNITY EXPERIENCE}
 \begin{itemize}[leftmargin=-2pt]\itemsep -2pt
 \item[]\textbf{International Rescue Committee}\\Refugee Empowerment Volunteer (Focus on Computer Support/Literacy)\hfill (Jan 2017 - Present)
 \item[] \textbf{Explorations, UC Davis Undergraduate Research Journal} 
 \\ Managing Editor \hfill (Sept 2015 - Present)
 \\Editor \hfill(Feb - June 2015)

\item[] \textbf{Neuroscience Initiative to Enhance Diversity} \\ Student Organizer \hfill (Event held April 2016)
\item[] \textbf {Neurobiology (class of 200 undergraduates)} \\Teaching Assistant \hfill (April - June 2015)


%, Davis, CA\\{\sl Editor} \hfill February 2015 - June 2015 \\
%{\sl Managing Editor, Social Sciences} \hfill September 2015 - Present
%\begin{itemize}\itemsep -2pt
%\item Communicate faculty reviewer comments to student authors
%\item Assess reviewed and edited papers for acceptance in a yearly UC Davis publication
%\item Copy-edit student articles
%\item Format articles for publication
%\end{itemize}

%\item[] \textbf{Women in Science and Engineering Mentorship Program} Mentor (Sept 2015-Present)

%\item[]\textbf{Neuroimaging Journal Club} Student Facilitator (Sept 2013 - Present)

%\begin{itemize}
%\item Invite faculty guests, select papers and lead student discussions for the student-run neuroimaging journal club
%\end{itemize}

%\item[]\textbf{Brain Awareness Week} Graduate Student Presenter (March 2013 - Present)

%\begin{itemize}
%\item Developed and presented a new station on Vestibular Nystagmus for a school for struggling high schoolers in Sacramento
%\item Presented posters at a booth at the Farmer's Market
%\item Taught neuroanatomy using sheep brains to high schoolers in Davis
%\end{itemize}

%\item[]\textbf{UC - Davis Neuroblog}, Graduate Student Contributor (Sept 2013 - Present)

%\begin{itemize}
%\item Plan, write and publish posts about neuroscience, advocacy and related topics to a student run blog.
%\end{itemize}

%\item[]\textbf{The Graduate Academic Achievement and Advocacy Program} Graduate Student Volunteer and Mentor (Sept 2013 - June 2015)

%\begin{itemize}
%\item Advice and assist students from underrepresented minorities one on one on topics such as graduate school applications and science writing
%\item Plan and run workshops for students from underrepresented minorities on topics such as graduate school applications, research assistantships and gap years
%\item Mentor a student
%\end{itemize}

%\item[]\textbf{Graduate Student Assembly} Departmental Representative (Sept 2013 - June 2015)
%\begin{itemize}
%\item Attend meetings and vote on policy affecting graduate students at UC Davis and in the wider UC graduate student assembly
%\item Publicize information to graduate group
%\end{itemize}

\end{itemize}
   
% \section{COMPUTER\\SKILLS} Windows, MacOS, Linux, Microsoft Office Suite, Internet, SPSS, E-Prime, SPM, MATLAB, Scheme, OCaml, Java,  \LaTeX\, R, DataGraph, FileMaker, Python, SQL




\end{resume}

\end{document}







